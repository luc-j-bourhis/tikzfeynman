\documentclass{article}
\usepackage{tikzfeynman}

\begin{document}

\begin{figure}
\centering
\begin{tikzpicture}[node distance=1cm]
\coordinate[vertex] (v1);
\coordinate[vertex, right=of v1] (v2);
\coordinate[vertex, right=of v2] (v3);
\coordinate[vertex, right=of v3] (v4);
\coordinate[above right=of v4, label=right:$f$] (f1);
\coordinate[below right=of v4, label=right:$f$] (f2);
\coordinate[above left =of v1, label=left :$e^+$] (e1);
\coordinate[below left =of v1, label=left :$e^-$] (e2);
\draw[fermion] (e1) -- (v1);
\draw[fermion] (v1) -- (e2);
\draw[photon] (v1) -- node[midway, above=0.1cm]{$Z$} (v2);
\draw[photon] (v3) -- node[midway, above=0.1cm]{$Z$} (v4);
\draw[fermion] (f1) -- (v4);
\draw[fermion] (v4) -- (f2);
\draw[fermion] (v2) to[semiloop] node[midway, above=0.1cm]{$q$} (v3);
\draw[fermion] (v3) to[semiloop] node[midway, below=0.1cm]{$q$} (v2);
\end{tikzpicture}
\caption{Demonstrating fermion lines, fermion loops, gauge bosons.}
\end{figure}

\begin{figure}
\centering
\begin{tikzpicture}[node distance=1cm,line width=1.2]
\coordinate[vertex] (v1);
\coordinate[vertex, right=of v1] (v2);
\coordinate[vertex, right=of v2] (v3);
\coordinate[vertex, right=of v3] (v4);
\coordinate[above right=of v4, label=right:$f$] (f1);
\coordinate[below right=of v4, label=right:$\bar{f}$] (f2);
\coordinate[above left =of v1, label=left :$e^+$] (e1);
\coordinate[below left =of v1, label=left :$e^-$] (e2);
\draw[fermion] (e1) -- (v1);
\draw[fermion] (v1) -- (e2);
\draw[photon,segment length=3mm] (v1) -- node[midway, above=0.1cm]{$Z$} (v2);
\draw[photon] (v3) -- node[midway, above=0.1cm]{$Z$} (v4);
\draw[fermion] (f1) -- (v4);
\draw[fermion] (v4) -- (f2);
\draw[fermion] (v2) to[semiloop] node[midway, above=0.1cm]{$q$} (v3);
\draw[fermion] (v3) to[semiloop] node[midway, below=0.1cm]{$q$} (v2);
\end{tikzpicture}
\caption{The same diagram again but tweaking some options: the ``wavelength'' of one $Z$ propagator, and the thickness of all lines.}
\end{figure}

\begin{figure}
\centering
\begin{tikzpicture}[node distance=1cm]
\coordinate[vertex] (v1);
\coordinate[vertex, right=of v1] (v2);
\coordinate[vertex, right=of v2] (v3);
\coordinate[vertex, right=of v3] (v4);
\coordinate[above right=of v4,label=right:$q$] (f1);
\coordinate[below right=of v4,label=right:$\bar{q}$] (f2);
\coordinate[above left =of v1,label=left :$\bar{q}$] (e1);
\coordinate[below left =of v1,label=left :$q$] (e2);
\draw[fermion] (e1) -- (v1);
\draw[fermion] (v1) -- (e2);
\draw[gluon] (v1) -- (v2);
\draw[gluon] (v3) -- (v4);
\draw[fermion] (f1) -- (v4);
\draw[fermion] (v4) -- (f2);
\draw[fermion] (v2) to[semiloop] node[midway, above=0.1cm]{$q$} (v3);
\draw[fermion] (v3) to[semiloop] node[midway, below=0.1cm]{$q$} (v2);
\end{tikzpicture}
\caption{Demonstrating gluon lines.}
\end{figure}

\begin{figure}
\centering
\begin{tikzpicture}[node distance=1cm]
\coordinate[vertex] (v1);
\coordinate[vertex, right=of v1] (v2);
\coordinate[vertex, right=of v2] (v3);
\coordinate[vertex, right=of v3] (v4);
\coordinate[above right=of v4,label=right:$q$] (f1);
\coordinate[below right=of v4,label=right:$\bar{q}$] (f2);
\coordinate[above left =of v1,label=left :$\bar{q}$] (e1);
\coordinate[below left =of v1,label=left :$q$] (e2);
\draw[fermion] (e1) -- (v1);
\draw[fermion] (v1) -- (e2);
\draw[gluon] (v1) -- (v2);
\draw[gluon] (v3) -- (v4);
\draw[fermion] (f1) -- (v4);
\draw[fermion] (v4) -- (f2);
\draw[gluon] (v2) to[semiloop] (v3);
\draw[gluon] (v3) to[semiloop] (v2);
\end{tikzpicture}
\caption{Demonstrating gluon loops.}
\end{figure}

\begin{figure}
\centering
\begin{tikzpicture}[node distance=1cm]
\coordinate[vertex] (v1);
\coordinate[vertex, right=2cm of v1] (v2);
\coordinate[vertex, above right=of v2] (v3);
\coordinate[vertex, above right=of v3] (v4);
\coordinate[above right=of v4,label=right:$q$] (f1);
\coordinate[below right=of v2,label=right:$\bar{q}$] (f2);
\coordinate[above left =of v1,label=left :$e^+$] (e1);
\coordinate[below left =of v1,label=left :$e^-$] (e2);
\draw[fermion] (e1) -- (v1);
\draw[fermion] (v1) -- (e2);
\draw[photon]  (v1) -- (v2);
\draw[fermion] (v3) -- (v2);
\draw[fermion] (f1) -- (v4);
\draw[fermion] (v2) -- (f2);
\draw[gluon] (v3) to[semiloop] (v4);
\draw[gluon] (v4) to[semiloop] (v3);
\end{tikzpicture}
\caption{Demonstrating more complex layout.}
\end{figure}

\begin{figure}
\centering
\begin{tikzpicture}[node distance=1cm]
\coordinate[vertex] (v1);
\coordinate[vertex, right=2cm of v1] (v2);

\coordinate[vertex, above left =of v1] (v1a);
\coordinate[vertex, below left =of v1] (v1b);
\coordinate[above left =of v1a, label=left :$e^+$] (ea);
\coordinate[below left =of v1b, label=left :$e^-$] (eb);

\coordinate[vertex, above right =of v2] (v2a);
\coordinate[vertex, below right =of v2] (v2b);
\coordinate[above right =of v2a, label=right :$q$] (fa);
\coordinate[below right =of v2b, label=right :$\bar{q}$] (fb);

\draw[fermion] (ea) -- (v1a);
\draw[fermion] (v1a) -- (v1);
\draw[fermion] (v1) -- (v1b);
\draw[fermion] (v1b) -- (eb);

\draw[fermion] (fa) -- (v2a);
\draw[fermion] (v2a) -- (v2);
\draw[fermion] (v2) -- (v2b);
\draw[fermion] (v2b) -- (fb);

\draw[photon]  (v1) -- (v2);

\draw[photon] (v1a) to[arcloop, centre=v1] node[midway, left=0.1cm]{$\gamma$} (v1b);
\draw[gluon]  (v2a) to[arcloop, centre=v2] node[midway, right=0.1cm]{g} (v2b);
\end{tikzpicture}
\caption{Demonstrating arcs.}
\end{figure}

\begin{figure}
\centering
\begin{tikzpicture}[node distance=1.5cm]
\coordinate[vertex] (v1);
\coordinate[vertex, right=of v1] (v2);
\coordinate[vertex, right=of v2] (v3);
\coordinate[vertex, right=of v3] (v4);
\draw[ghost] (v1) -- (v2);
\draw[ghost] (v2) -- (v3);
\draw[ghost] (v3) -- (v4);
\draw[gluon] (v2) to[semiloop] (v3);
\end{tikzpicture}
\caption{Demonstrating 180$^\circ$ arc and ghost lines.}
\end{figure}

\begin{figure}
\centering
\begin{tikzpicture}[node distance=2cm]
\coordinate[vertex] (v1);
\coordinate[vertex, right=of v1] (v2);
\coordinate[right=1cm of v2] (o);
\coordinate[vertex, right=1cm of o] (v3);
\coordinate[vertex, right=of v3] (v4);
\coordinate[vertex, above=1cm of o] (v5);
\coordinate[vertex, below=1cm of o] (v6);
\draw[gluon] (v1) -- (v2);
\draw[gluon] (v3) -- (v4);
\draw[ghost] (v2) to[arcloop,centre=o] (v6);
\draw[ghost] (v6) to[arcloop,centre=o] (v3);
\draw[ghost] (v3) to[arcloop,centre=o] (v5);
\draw[ghost] (v5) to[arcloop,centre=o] (v2);
\draw[gluon] (v5) -- (v6);
\end{tikzpicture}
\caption{Demonstrating ghost loops with gluon insertions.}
\end{figure}

\begin{figure}
\centering
\begin{tikzpicture}[node distance=1.5cm]
\coordinate[vertex] (g1);
\coordinate[vertex, below=of g1] (g2);
\coordinate[vertex, right=of g1] (vgtt1);
\coordinate[vertex, below=of vgtt1] (vgtt2);
\draw[gluon] (g1) -- (vgtt1);
\draw[gluon] (g2) -- (vgtt2);
\draw[fermion] (vgtt1) -- (vgtt2);
\draw[fermion] (vgtt2) -- +(30:1.5cm) coordinate[vertex] (vtth1);
\draw[fermion] (vtth1) -- (vgtt1);

\draw[scalar] (vtth1) -- node[midway, above=0.1cm]{$H^0$} +(0:2) coordinate[vertex] (vtth2);

\draw[fermion] (vtth2) -- +(-30:1.5cm) coordinate[vertex] (vttgam1);
\coordinate[vertex, above=of vttgam1] (vttgam2);
\draw[fermion] (vttgam1) -- (vttgam2);
\draw[fermion] (vttgam2) -- (vtth2);
\coordinate[vertex, right=of vttgam1] (gam1);
\coordinate[vertex, right=of vttgam2] (gam2);
\draw[photon] (vttgam1) -- (gam1);
\draw[photon] (vttgam2) -- (gam2);
\end{tikzpicture}
\caption{Demonstrating scalar.}
\end{figure}

\end{document}
